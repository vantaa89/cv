\documentclass{cv}
\usepackage[utf8]{inputenc}

\title{cv}
\setname{SEOJUNE LEE}
\setaltname{이서준}
\setmail{leeseojune@snu.ac.kr}
\setphone{(+82) 10-7643-2441}

\usepackage[]{hyperref} % remove color from links

\begin{document}

\cvheader

%%%%%%%%%%%%%%%%%%%%%%%%%%%%%%
%%% INTERESTS
%%%%%%%%%%%%%%%%%%%%%%%%%%%%%%

\cvlargeline{Interests}{Biomedical Imaging, Deep Learning}


%%%%%%%%%%%%%%%%%%%%%%%%%%%%%%
%%% EDUCATION
%%%%%%%%%%%%%%%%%%%%%%%%%%%%%%

\cvsection{Education}

\cventry{Seoul National University}{Seoul, Korea}
{Undergraduate Student in Electrical and Computer Engineering}{Mar. 2021 - Present}
{
    \item Minor in Linguistics
    \item GPA: Overall 4.26/4.3, Major 4.17/4.3, Minor 4.30/4.30
}
\cventry{\href{http://ksa.hs.kr/}{Korea Science Academy of KAIST}}{Busan, Korea}
{High School}{Feb. 2018 - Feb. 2021}
{
    \item GPA: 4.17/4.3 (Rank: 4/131) 
    \item Graduated with Distinction in Physics (2\textsuperscript{nd} place)
}


%%%%%%%%%%%%%%%%%%%%%%%%%%%%%%
%%% RESEARCH PROJECTS
%%%%%%%%%%%%%%%%%%%%%%%%%%%%%%
\cvsection{Research Experiences}
\cventry{Laboratory of Imaging Science and Technology (LIST)}{Seoul National University}
{Undergraduate Research Intern}{Jun. 2022 - Aug. 2022}
{
    \item Advised by professor Jongho Lee
    \item Studied detection and correction of motion artifact of magnetic resonance images
}

\cventry{Research \& Education Program (R\&E)}{Korea Science Academy}
{Title: On Wave Propagation in Hyperhelix Structures}{Mar. 2019 - Dec. 2019}
{
    \item Advised by Dr. Yongdeok Kim
    \item Implemented a mechanical wave simulator for curved waveguide using python %\href{https://research.ksa.hs.kr/2019RNE_MAT02}{[paper]}
    \item Gave a poster presentation at International Science Youth Forum (ISYF) @ Singapore 2020
}

%%%%%%%%%%%%%%%%%%%%%%%%%%%%%%
%%% HONORS & SCHOLARSHIPS
%%%%%%%%%%%%%%%%%%%%%%%%%%%%%%
\cvsection{Honors \& Scholarships}
% \cvhonor{Special Prize in SNU FastMRI Challenge, \textit{Seoul National University}}{2022}
\cvhonor{\textbf{The National Scholarship for Science and Engineering}, \textit{Korea Student Aid Foundation} (full tuition)}{2021}
\cvhonor{\textbf{Hanseong Nobel Scholarship for the Gifted}, \textit{Hanseong Sonjaehan Foundation} (\$10000 equivalent)} {2018}
\cvhonor{Bronze Prize in \textbf{Korea Olympiads in Informatics}, \textit{Ministry of Science and ICT}}{2018}

\vspace{-4mm}

%%%%%%%%%%%%%%%%%%%%%%%%%%%%%%
%%% TEACHING
%%%%%%%%%%%%%%%%%%%%%%%%%%%%%%
\cvsection{Teaching Experiences}
\cvhonor{\textbf{Linear Algebra for Electrical Systems}, SNU, Undergraduate Peer Tutor}{Fall 2022}

\vspace{-4mm}
%%%%%%%%%%%%%%%%%%%%%%%%%%%%%%
%%% SKILLS
%%%%%%%%%%%%%%%%%%%%%%%%%%%%%%
\cvsection{Skills}
\cvline{Programming}{C++, Python, MATLAB, Java, R}
\cvline{Tools}{Git, \LaTeX{}, PyTorch}

%%%%%%%%%%%%%%%%%%%%%%%%%%%%%%
%%% EXTRACURRICULAR
%%%%%%%%%%%%%%%%%%%%%%%%%%%%%%
\cvsection{Extracurricular Activities}
\cventry{\href{https://www.linkedin.com/company/outta아우타/}{OUTTA}}{Mar. 2022 - Aug. 2022}
{Student Mentor}{}
{
    \item Organized online deep learning bootcamp, gave lectures on natural language processing
}

%%%%%%%%%%%%%%%%%%%%%%%%%%%%%%
%%% MISCELLANIES
%%%%%%%%%%%%%%%%%%%%%%%%%%%%%%
\cvsection{Miscellanies}
\cvlinewide{Algorithmic Problem Solving}{\hspace{-10mm}Solved 600+ Problems at Baekjoon Online Judge \href{https://www.acmicpc.net/user/myju742}{[profile]}}
\cvline{Coursera}{\hspace{-15mm}Completed online specialization ``Generative Adversarial Networks'', \textit{DeepLearning.AI} \href{https://coursera.org/share/8a940d19ca1c4378a7db177d7c50a4be}{[certificate]}}
\cvline{English}{\hspace{-15mm}TOEIC: 970/990 (expired)}


\newpage
%%%%%%%%%%%%%%%%%%%%%%%%%%%%%%
%%% Relevant Coursework
%%%%%%%%%%%%%%%%%%%%%%%%%%%%%%
\cvsection{Relevant Coursework}
\cvcourse{2021-2}{Creative Engineering Design, Programming Methodology, Linear Algebra for Electrical Systems}
\cvcourse{2022-1}{Signals and Systems, Introduction to Circuit Theory and Laboratory, Computational Linguistics}
\vspace{3pt}
\cvcourse{2022-2}{Digital Logic Design \& Lab, Intro. Electromagnetism with Practice, System Programming, \\ Mathematical Foundations of Deep Neural Networks, Syntax}
\vspace{7pt}


\end{document}