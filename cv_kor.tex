\documentclass{cv}
\usepackage[utf8]{inputenc}

\title{cv}
\setname{이서준}
\setaltname{Seojune Lee}
\setmail{leeseojune@snu.ac.kr}
\setphone{(+82)10-7643-2441}


\date{March 2022}

\begin{document}

\cvheader

%%%%%%%%%%%%%%%%%%%%%%%%%%%%%%
%%% INTERESTS
%%%%%%%%%%%%%%%%%%%%%%%%%%%%%%

\cvlargeline{관심분야}{Biomedical Imaging, Natural Language Processing}


%%%%%%%%%%%%%%%%%%%%%%%%%%%%%%
%%% EDUCATION
%%%%%%%%%%%%%%%%%%%%%%%%%%%%%%

\cvsection{교육}

\cventry{\large{서울대학교}}{서울특별시}
{학사과정}{2021년 3월 - 현재}
{
    \item 전기$\cdot$정보공학부
    \item GPA: 4.24/4.3
}
\cventry{\href{http://ksa.hs.kr/}{\large{KAIST 부설 한국과학영재학교}}}{부산광역시}
{고등학교}{2018년 2월 - 2021년 2월}
{
    \item 부산광역시교육감 졸업우수 표창. GPA: 4.17/4.3
}

%%%%%%%%%%%%%%%%%%%%%%%%%%%%%%
%%% HONORS & AWARDS
%%%%%%%%%%%%%%%%%%%%%%%%%%%%%%

\cvsection{수상}

\textbf{국가우수이공계장학금}, \textit{한국장학재단}, 2021

동상, 한국정보올림피아드, \textit{과학기술정보통신부}, 2018

%%%%%%%%%%%%%%%%%%%%%%%%%%%%%%
%%% RESEARCH PROJECTS
%%%%%%%%%%%%%%%%%%%%%%%%%%%%%%
\cvsection{연구 프로젝트}
\cventry{초나선형 용수철에서의 파동 전파에 관한 연구}{2019}
{지도교사: 김용덕 박사}{한국과학영재학교}
{
    \item 팀장으로 활동해 도파로에서의 파동 진행을 분석할 수 있는 시뮬레이터를 구현하였습니다.
    \item NumPy와 SciPy 등 패키지와 멀티코어 컴퓨팅을 활용해보았습니다. (\href{https://research.ksa.hs.kr/2019RNE_MAT02}{논문})
}


%%%%%%%%%%%%%%%%%%%%%%%%%%%%%%
%%% SKILLS
%%%%%%%%%%%%%%%%%%%%%%%%%%%%%%
\cvsection{보유 기술}

\cvline{프로그래밍}{C++, Python, MatLab}
\cvline{툴}{Git, \LaTeX{}, PyTorch}


%%%%%%%%%%%%%%%%%%%%%%%%%%%%%%
%%% Relevant Coursework
%%%%%%%%%%%%%%%%%%%%%%%%%%%%%%
\cvsection{관련 수강과목}
\textbf{서울대학교}

\cvcourse{2021-2}{창의공학설계, 프로그래밍방법론, 전기시스템선형대수}
\cvcourse{2022-1}{신호 및 시스템, 기초회로이론 및 실험, 확률변수 및 확률과정의 기초, 컴퓨터언어학}

\textbf{KAIST 부설 한국과학영재학교}
\begin{itemize}
    \item 선형대수, 미분방정식, 수학적모델링, 미적분학3 (벡터미적분학), 자료구조, 정보과학세미나 (기계학습), 현대물리학개론
\end{itemize}

%%%%%%%%%%%%%%%%%%%%%%%%%%%%%%
%%% MISCELLANIES
%%%%%%%%%%%%%%%%%%%%%%%%%%%%%%
\cvsection{그 외}
\cvline{영어}{970/990 (TOEIC), 404/600 (TOEFL)}

\end{document}
